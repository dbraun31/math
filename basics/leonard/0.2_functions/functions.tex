\documentclass{article}
\usepackage{graphicx}
\usepackage{amsmath}
\usepackage{enumerate}
\usepackage{amsfonts}

\begin{document}

\title{Lecture 0.2: Functions}
\author{Professor Leonard}
\maketitle

This lecture is all about a review of functions.


\section{Functions}
The first big point he's making is that in order for something to be a function, each
input $X$ needs to be associated with only one output $f(X)$.

\subsection{Special Functions}

Need to be careful with something like the following:

\begin{align}
    x^2 + y^2 = 25\\
    y^2 = 25 - x^2\\
    y = \pm \sqrt{25 - x^2}
\end{align}

This isn't a function because there can always be two values (plus or minus) that satisfy
$y^2$.\\

\subsubsection{Piecewise functions}

One of the most common functions is the absolute value function, which is an example of a
\emph{piecewise function}:

\begin{equation}
    \lvert x \rvert = 
    \begin{cases}
        x & \text{if } x \geq 0\\
        -x & \text{if } x < 0
    \end{cases}
\end{equation}

\subsubsection{Domain and range}

\textbf{Domain:} All input values for a function.\\ 
\textbf{Range:} All output values for a function.\\

There can be physical restraints (eg, no negative distance), or formulaic restraints (eg,
$1/x~x \neq 0$.

\textbf{Natural Domain:} All values that 'work' in the function.\\

Example:


\begin{align*}
f(x) = x^3\\
x \in \mathbb{R}\\
\\
g(x) = \frac{1}{(x-1)(x-3)}\\
x \neq 1,~x \neq 3
\end{align*}

Need to watch out for denominators and roots.\\

He's giving the following example:

\begin{align}
    f(x) = \sqrt{x^2 - 5x + 6}\\
    x^2 - 5x + 6 \geq 0\\
    (x-2)(2-3) \geq 0
\end{align}

You can plot the quadratic, can only take the domain values that give a positive (or zero)
range. You wind up with:

\begin{equation}
    D: (-\infty, 2] \cup [3, \infty)
\end{equation}

\textbf{Interesting point.} Sometimes you can simplify a function. If you do so, which you
should, you need to keep the constraints of the original domain. For example:

\begin{align}
    f(x) = \frac{x^2-4}{x-2}\\
    2 \notin D
\end{align}

Even though $f(x)$ can be simplified

\begin{align}
    f(x) = \frac{x^2 - 4}{x - 2}\\
    f(x) = \frac{(x + 2)(x - 2)}{x - 2}\\
    f(x) = x + 2
\end{align}

You would plot such a function with a hole in the graph. If you can simplify and cancel
out the restriction, it's a hole. If you can't cancel it out, it's an asymptote.\\

To figure out restrictions on the range, first find restrictions on domain, then see what
kind of interval you have on the range. Another way is to rearrange the function to solve
for the outcome and then observe the restrictions on the range. Eg

\begin{align}
    y = \frac{x + 1}{x - 1}\\
    x = \frac{y + 1}{y - 1}
\end{align}

Saying this method won't work for every function, however.\\

The other restrictions you need to be mindful of come in the context of real life examples
(eg, can't have negative length).


\subsubsection{Odd and even functions}

Even functions are symmetric around the y axis $f(-x) = f(x)$. Odd functions are symmetric around the
origin $f(-x) = -f(x)$.

\end{document}

