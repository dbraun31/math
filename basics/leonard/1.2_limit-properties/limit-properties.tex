\documentclass{article}
\usepackage{graphicx}
\usepackage{float}
\usepackage{amsmath}
\usepackage{enumerate}
\usepackage{amsfonts}

\setlength{\parindent}{0pt}

\begin{document}

\title{Lecture 1.2: Limit Properties. Techniques of Limit Computation}
\author{Professor Leonard}
\maketitle

\section{Basics}

\begin{enumerate}
    \item \textbf{Limit of a constant}
$$
\lim_{x \to a} C = C
$$
The limit of a constant is always the constant (think because the plot is just a
horizontal line).

    \item \textbf{Limit of $x$}
        $$
        \lim_{x\to a} x = a
        $$
        Think because $f(x) = x$ is just identity, so point $a$ on $x$ is also $a$ on $y$.

    \item \textbf{Zero limit}
        \begin{align*}
            \lim_{x\to 0^-} \frac{1}{x}= -\infty\\
            \lim_{x\to 0^+} \frac{1}{x} = \infty
        \end{align*}

\end{enumerate}

\section{Properties}

Given two functions with limits that exist:
\begin{align*}
    \lim{x\to a}f(x) = L_1\\
    \lim{x\to a} g(x) = L_2
\end{align*}

\begin{enumerate}
    \item \textbf{Function joining and separation}
        $$
        \lim_{x\to a} [f(x) \pm g(x)] = \lim_{x\to a}f(x) \pm \lim_{x\to a}g(x)
        $$
        This operation works both ways (can also join).

    \item \textbf{Same thing for multiplication}
        $$
        \lim_{x\to a} [f(x) \cdot g(x)] = \lim_{x\to a}f(x) \cdot \lim_{x\to a}g(x)
        $$
    \item \textbf{For division}
        $$
        \lim_{x\to a} \left[ \frac{f(x)}{g(x)} \right] = \frac{\lim_{x\to
        a}f(x)}{\lim_{x\to a}g(x)}, \lim_{x\to a}g(x) \neq 0
        $$

    \item \textbf{Exponents}
        $$
        \lim_{x\to a}[f(x)]^n = \left[\lim_{x\to a} f(x) \right]^n \rightarrow \lim_{x\to
        a} \sqrt[n]{f(x)} \rightarrow \sqrt[n]{\lim_{x\to a}f(x)}
        $$


\end{enumerate}

\end{document}
