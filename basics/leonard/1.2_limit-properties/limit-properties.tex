\documentclass{article}
\usepackage{graphicx}
\usepackage{float}
\usepackage{amsmath}
\usepackage{enumerate}
\usepackage{amsfonts}

\setlength{\parindent}{0pt}

\begin{document}

\title{Lecture 1.2: Limit Properties. Techniques of Limit Computation}
\author{Professor Leonard}
\maketitle

\section{Basics}

\begin{enumerate}
    \item \textbf{Limit of a constant}
$$
\lim_{x \to a} C = C
$$
The limit of a constant is always the constant (think because the plot is just a
horizontal line).

    \item \textbf{Limit of $x$}
        $$
        \lim_{x\to a} x = a
        $$
        Think because $f(x) = x$ is just identity, so point $a$ on $x$ is also $a$ on $y$.

    \item \textbf{Zero limit}
        \begin{align*}
            \lim_{x\to 0^-} \frac{1}{x}= -\infty\\
            \lim_{x\to 0^+} \frac{1}{x} = \infty
        \end{align*}

\end{enumerate}

\section{Properties}

Given two functions with limits that exist:
\begin{align*}
    \lim{x\to a}f(x) = L_1\\
    \lim{x\to a} g(x) = L_2
\end{align*}

\begin{enumerate}
    \item \textbf{Function joining and separation}
        $$
        \lim_{x\to a} [f(x) \pm g(x)] = \lim_{x\to a}f(x) \pm \lim_{x\to a}g(x)
        $$
        This operation works both ways (can also join).

    \item \textbf{Same thing for multiplication}
        $$
        \lim_{x\to a} [f(x) \cdot g(x)] = \lim_{x\to a}f(x) \cdot \lim_{x\to a}g(x)
        $$
    \item \textbf{For division}
        $$
        \lim_{x\to a} \left[ \frac{f(x)}{g(x)} \right] = \frac{\lim_{x\to
        a}f(x)}{\lim_{x\to a}g(x)}, \lim_{x\to a}g(x) \neq 0
        $$

    \item \textbf{Exponents}
        $$
        \lim_{x\to a}[f(x)]^n = \left[\lim_{x\to a} f(x) \right]^n \rightarrow \lim_{x\to
        a} \sqrt[n]{f(x)} \rightarrow \sqrt[n]{\lim_{x\to a}f(x)}
        $$
\subsection{Example}

\begin{align*}
    \lim_{x \to 2} (x^3 - 2x + 7) &\rightarrow \lim_{x \to 2} x^3 - \lim_{x \to 2} 2x +
\lim_{x\to 2} 7\\
                                  &\rightarrow \left[ \lim_{x \to 2} x \right]^3 - \left(
                                  \lim_{x\to 2} 2\right) \cdot \left( \lim_{x\to 2} x
                              \right) + \lim_{x\to 2} 7\\
                                  & \rightarrow 2^3 - 2 \cdot 2 + 7\\
                                  & \rightarrow 11\\
    f(2) &= 11
\end{align*}

For any polynomial, all you need to do is plug the value at the limit as the variable in
order to determine the limit. The general case for any $P$ polynomial:

$$
\lim_{x \to a}P(x) = P(a)
$$

\subsection{Holes and asymptotes}

\subsubsection{Holes}
$$
\lim_{x\to 2} \frac{x^2 - 4}{x-2}
$$

With limits, because you're not actually reaching the point at the limit, it's okay to
simplify out a domain problem. So:

\begin{align*}
    \lim_{x\to 2} &\frac{x^2 -4}{x-2}\\
                  & \frac{(x+2)(x-2)}{x-2}\\
                  & x+2\\
    \lim_{x\to 2} &= 4
\end{align*}

If the domain issue can simplify, it's a hole in the graph. If it can't, then it's some
type of asymptote. If a fraction reduces to zero over zero, I think he's saying that means
it can be factored.

\subsubsection{Asymptotes}

\begin{align*}
    \lim_{x \to 5} \frac{x^3 - 3x - 10}{x^2 - 10x + 25}\\
    \lim_{x \to 5} \frac{(x-5)(x+2)}{(x-5)(x-5)}\\
    \lim_{x \to 5} \frac{x+2}{x-5}
\end{align*}

We have ourselves a vertical asymptote.

\textbf{Sign analysis test}\\

To determine the direction of the asymptotes in both directions, we need to construct a
number line between the two values that will make the numerator and denominator equal to
zero (here $-2$ and $5$). Sample on both sides of the limit, plug into the function, and
determine the sign on both sides.


    
Left off at 47:38.


\end{enumerate}

\end{document}
