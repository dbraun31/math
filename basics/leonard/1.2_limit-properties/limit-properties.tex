\documentclass{article}
\usepackage{graphicx}
\usepackage{float}
\usepackage{amsmath}
\usepackage{enumerate}
\usepackage{amsfonts}
\usepackage{tikz}

\setlength{\parindent}{0pt}

\begin{document}

\title{Lecture 1.2: Limit Properties. Techniques of Limit Computation}
\author{Professor Leonard}
\maketitle

\section{Basics}

\begin{enumerate}
    \item \textbf{Limit of a constant}
$$
\lim_{x \to a} C = C
$$
The limit of a constant is always the constant (think because the plot is just a
horizontal line).

    \item \textbf{Limit of $x$}
        $$
        \lim_{x\to a} x = a
        $$
        Think because $f(x) = x$ is just identity, so point $a$ on $x$ is also $a$ on $y$.

    \item \textbf{Zero limit}
        \begin{align*}
            \lim_{x\to 0^-} \frac{1}{x}= -\infty\\
            \lim_{x\to 0^+} \frac{1}{x} = \infty
        \end{align*}

\end{enumerate}

\section{Properties}

Given two functions with limits that exist:
\begin{align*}
    \lim_{x\to a}f(x) = L_1\\
    \lim_{x\to a} g(x) = L_2
\end{align*}

\begin{enumerate}
    \item \textbf{Function joining and separation}
        $$
        \lim_{x\to a} [f(x) \pm g(x)] = \lim_{x\to a}f(x) \pm \lim_{x\to a}g(x)
        $$
        This operation works both ways (can also join).

    \item \textbf{Same thing for multiplication}
        $$
        \lim_{x\to a} [f(x) \cdot g(x)] = \lim_{x\to a}f(x) \cdot \lim_{x\to a}g(x)
        $$
    \item \textbf{For division}
        $$
        \lim_{x\to a} \left[ \frac{f(x)}{g(x)} \right] = \frac{\lim_{x\to
        a}f(x)}{\lim_{x\to a}g(x)}, \lim_{x\to a}g(x) \neq 0
        $$

    \item \textbf{Exponents}
        $$
        \lim_{x\to a}[f(x)]^n = \left[\lim_{x\to a} f(x) \right]^n \rightarrow \lim_{x\to
        a} \sqrt[n]{f(x)} \rightarrow \sqrt[n]{\lim_{x\to a}f(x)}
        $$

\end{enumerate}

\subsection{Example}

\begin{align*}
    \lim_{x \to 2} (x^3 - 2x + 7) &\rightarrow \lim_{x \to 2} x^3 - \lim_{x \to 2} 2x +
\lim_{x\to 2} 7\\
                                  &\rightarrow \left[ \lim_{x \to 2} x \right]^3 - \left(
                                  \lim_{x\to 2} 2\right) \cdot \left( \lim_{x\to 2} x
                              \right) + \lim_{x\to 2} 7\\
                                  & \rightarrow 2^3 - 2 \cdot 2 + 7\\
                                  & \rightarrow 11\\
    f(2) &= 11
\end{align*}

For any polynomial, all you need to do is plug the value at the limit as the variable in
order to determine the limit. The general case for any $P$ polynomial:

$$
\lim_{x \to a}P(x) = P(a)
$$

\subsection{Holes and asymptotes}

\subsubsection{Holes}
$$
\lim_{x\to 2} \frac{x^2 - 4}{x-2}
$$

With limits, because you're not actually reaching the point at the limit, it's okay to
simplify out a domain problem. So:

\begin{align*}
    \lim_{x\to 2} &\frac{x^2 -4}{x-2}\\
                  & \frac{(x+2)(x-2)}{x-2}\\
                  & x+2\\
    \lim_{x\to 2} &= 4
\end{align*}

If the domain issue can simplify, it's a hole in the graph. If it can't, then it's some
type of asymptote. If a fraction reduces to zero over zero, I think he's saying that means
it can be factored.

\subsubsection{Asymptotes}

\begin{align*}
    \lim_{x \to 5} \frac{x^3 - 3x - 10}{x^2 - 10x + 25}\\
    \lim_{x \to 5} \frac{(x-5)(x+2)}{(x-5)(x-5)}\\
    \lim_{x \to 5} \frac{x+2}{x-5}
\end{align*}

We have ourselves a vertical asymptote.

\textbf{Sign analysis test}

To determine the direction of the asymptotes in both directions, we need to construct a
number line between the two values that will make the numerator and denominator equal to
zero (here $-2$ and $5$). Sample on both sides of the limit, plug into the function, and
determine the sign on both sides.\\

(coming back after awhile; I think the below is an example of factoring a limit, not an
example of the sign analysis test...)\\

\textbf{Example.}

\begin{align}
    \lim_{x \to 1} \frac{x-1}{\sqrt{x} - 1}\\
    \lim_{x \to 1} \frac{x-1}{\sqrt{x} - 1} \cdot \frac{\sqrt{x} + 1}{\sqrt{x} + 1}\\
    \lim_{x \to 1} \frac{(x - 1)(\sqrt{x} + 1)}{x - 1}\\
    \lim_{x \to 1} \sqrt{x} + 1\\
    \lim_{x \to 1} \sqrt{1} + 1 = 2
\end{align}

The logic behind (2) is to multiply by the \emph{conjugate}, since you can't necessarily
factor anything. 

The trick to get the denominator in step (3) is:

\begin{align*}
    (\sqrt{x} - 1) \cdot (\sqrt{x} + 1)\\
    \sqrt{x} \cdot \sqrt{x} + \sqrt{x} - \sqrt{x} - 1\\
    x - 1
\end{align*}

Forget what the formal name of that is... but it's the same way you'd distribute one term
into another to get a classic polynomial. Been a minute. Yea so conjugate is the term you
multiply another term by to get that middle term to cancel. Eg, the conjugate of $p-q$ is
$p+q$.

\textbf{Another example.}

\begin{align*}
    \lim_{x \to 0} \frac{\sqrt{1 + x} - 1}{x}\\
    \lim_{x\to 0} \frac{\sqrt{1+x} - 1}{x} \cdot \frac{\sqrt{1+x} + 1}{\sqrt{1+x} + 1}\\
    \lim_{x\to 0} \frac{x}{x(\sqrt{1+x} + 1)}\\
    \lim_{x \to 0} \frac{1}{\sqrt{1 + x} + 1}\\
    \lim_{x\to 0} \frac{1}{\sqrt{1} + 1}\\
    \lim_{x\to 0} \frac{1}{2}
\end{align*}

\section{Piece-Wise Limits}

We're going to find some one-sided limits and see if they are equal. \\

\textbf{Example.}

$$
f(x) = \begin{cases}
    \frac{1}{x+2} & \text{if } x < -2\\
    x^2 - 5 & \text{if } -2 > x \leq 3\\
    \sqrt{x+13} & \text{if } x > 3
\end{cases}
$$
A helpful approach is to plot a number line and draw the intervals of the piece-wise
function:\\

\begin{tikzpicture}
    \draw[-] (-4, 0) -- (4, 0);
    \draw[-] (-2, -.1) -- (-2, .1);
    \draw[-] (3, -.1) -- (3, .1);

    \node[below] at (-2, -0.1) {$-2$};
    \node[below] at (3, -0.1) {$3$};
    \node[above] at (-3, 0.1) {$f_1$};
    \node[above] at (0, 0.1) {$f_2$};
    \node[above] at (4, .1) {$f_3$};

\end{tikzpicture}

If the functions have the same limit as they approach the interval between them, then the
limit can be said to exist; otherwise, it doesn't exist.\\

Let's work through the first two functions with $-2$ as the limit:

\begin{align*}
    f_1(x) &= \lim_{x \to -2} \frac{1}{x + 2}\\
           &= \lim_{x\to -2} \frac{1}{x+2} \cdot \frac{x-2}{x-2}\\
           &= \lim_{x\to -2} \frac{x-2}{x^2-4}
\end{align*}

Hmm okay I'm stuck. Multiplying by the conjugate doesn't seem to free up the denominator
from being zero when plugging in the limit.\\

Aha okay so if the function doesn't reduce to $0/0$, that suggests there's an asymptote,
not a hole. Then we need to do the sign analysis test to determine the direction of the
asymptote.

\begin{align*}
    f_1(x) &= \lim_{x\to -2} \frac{1}{-3+2}\\
           &= -1\\
    \lim_{x\to -2^-} \frac{1}{x + 2} &= -\infty
\end{align*}

Because the limit approaching $-2$ from the left does not equal the limit approaching $-2$
from the right (I didn't show computation for approaching from the right here), the limit
is said to not exist.

\section{Limits of Trig Fucnctions}

First thing to note is that $sin(x)$ and $cos(x)$ are continuous everywhere.

\begin{align*}
    \lim_{x\to a} sin(x) &= sin(a)\\
    \lim_{x\to a} cos(x) &= cos(a)\\
    \lim_{x\to a} tan(x) &= \lim_{x\to a} \frac{sin(x)}{cos(x)}\\
                         &= \frac{\lim_{x\to a} sin(x)}{\lim_{x\to a} cos(x)}\\
                         &= \frac{sin(a)}{cos(a)}\\
    \lim_{x\to a} tan(x) &= tan(a), cos(x) \neq 0, x \neq \pm \frac{\pi}{2}
\end{align*}

\textbf{Example.}

\begin{align}
    \lim_{x\to 1} cos \left( \frac{x^2-1}{x-1}\right)\\
    cos \left[ \lim_{x\to 1} \frac{x^2-1}{x-1} \right]\\
    cos \left[ \lim_{x\to 1} \frac{(x+1)(x-1)}{x-1} \right]\\
    cos[\lim_{x\to 1} x+1]\\
    cos(2)
\end{align}

I guess (7) is a legal move. His justification was "Cosine is continuous, so by
composition...".\\

\textbf{Example.}

\begin{align*}
    \lim_{x\to \pi/2} [3x^2 + cos~x]\\
    \lim_{x\to \pi/2} 3 \cdot \left( \frac{\pi}{2} \right)^2 + 0\\
    \lim_{x \to \pi/2} 3 \cdot \left( \frac{\pi^2}{4} \right)\\
    \lim_{x \to \pi/2} \frac{3\pi^2}{4}\\
    \lim_{x\to \pi/2} [3x^2 + cos~x] = \frac{3\pi^2}{4}
\end{align*}

Checking against his solution. After substituting the limit in for $x$ (step 2), you can
drop the limit from the computations. Otherwise everything is good here.\\

\textbf{Example.}

\begin{align*}
    \lim_{x\to 0} \frac{sin(x)}{x}
\end{align*}

He goes through a crazy procedure of trig and the unit circle to eventually derive $1 \geq
sin(x) / x \geq cos(x)$ and demonstrates the \textbf{squeeze theorem}, which says that
when a function falls between two functions that share a limit, then the squeezed function
also shares that limit. \\

\emph{Note:} When you take the reciprocal when inequalities are involved you need to flip
the inequality.\\

Alright so I guess there are a few trig limits (ie, identities) that are important to
know. They are the following:

\begin{align*}
    \lim_{x\to 0} \frac{sin(x)}{x} &= 1 \\
    \lim_{x\to 0} \frac{1 - cos(x)}{x} &= 0\\
    \lim_{x\to 0} \frac{tan(x)}{x} &= 1
\end{align*}

These all have elaborate proofs that he walked through that I'm not reproducing here. These will all be important to use when solving for other limits.

\textbf{Example.}

\begin{align*}
    \lim_{x\to 0} \frac{sin(2x)}{x}\\
    \lim_{x\to 0} \frac{sin(2x)}{x} \cdot \frac{2}{2}\\
    2 \cdot \lim_{x\to 0} \frac{sin(2x)}{2x} \\ 
    u = 2x \\
    2 \cdot \lim_{x\to 0} \frac{sin(u)}{u}\\
    2 \cdot 1 = 2
\end{align*}

Making a substitution with $u = 2x$ because as $2x$ approaches $0$ so does $u$.\\

\textbf{Example.}

\begin{align*}
    \lim_{x\to 0} \frac{sin(5x)}{sin(6x)}\\
    \lim_{x\to 0} \frac{sin(5x)}{sin(6x)} \cdot \frac{\frac{1}{x}}{\frac{1}{x}}\\
    \lim_{x\to 0} \frac{\frac{sin(5x)}{x}}{\frac{sin(6x)}{x}}\\
    \frac{5 \cdot \lim_{x\to 0} \frac{sin(5x)}{5x}}{6 \cdot \lim_{x\to 0}
    \frac{sin(6x)}{6x}}\\
    \frac{5 \cdot 1}{6 \cdot 1}\\
    \frac{5}{6}
\end{align*}


\textbf{Example.}\\

\begin{align*}
    \lim_{x\to 0} \frac{sin(x^2)}{x}\\
    \lim_{x\to 0} \frac{sin(x^2)}{x} \cdot \frac{x}{x}\\
    \lim_{x\to 0} \frac{x \cdot sin(x^2)}{x^2}\\
    \lim_{x\to 0} x \cdot \lim_{x\to 0} \frac{sin(x^2)}{x^2}\\
    0 \cdot 1 = 0
\end{align*}

I could really use a refesher on the algebraic rules on how you can move things around
when multiplying fractions.\\

Ah so in the earlier examples when he pulled out a constant and dropped the limit
notation, that's because the limit as $x$ approaches any number when the function is a
constant is just that constant (ie, $\lim_{x\to 0} 2 = 2$).\\

Left off at 2:24

\end{document}

