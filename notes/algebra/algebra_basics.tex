\documentclass{article}
\usepackage{amsmath}
\setlength{\parindent}{0pt}

\title{Algebra Basic Rules}
\author{Dave Braun}

\begin{document}
\maketitle

This is a document where I want to list some of the basic but tricky algebra fundamental
rules that I've probably forgotten about over the years. I'll keep adding to this list as
I go over more complicated stuff and run into things that I've forgotten about.

\section{Fraction-Fu}

\textbf{Addition}\\
Need a common denominator.

\begin{align*}
    \frac{1}{2} + \frac{3}{4}\\
    \frac{2}{4} + \frac{3}{4}\\
    \frac{5}{4}
\end{align*}

\textbf{Multiplication}\\
Don't need a common denominator.

\begin{align*}
    \frac{1}{2} \cdot \frac{3}{4}\\
    \frac{1 \cdot 3}{2 \cdot 4}
\end{align*}

\textbf{Division}\\
Multiply by the reciprocal of the denominator.

\begin{align*}
    \frac{\frac{1}{2}}{\frac{3}{4}}\\
    \frac{1}{2} \cdot \frac{3}{4}\\
    \frac{3}{8}
\end{align*}

\textbf{Splitting a constant}\\
If there's a constant separated by addition and subtraction, you can split that term so
long as the overall value stays the same.

$$
\frac{2 - x - y}{z} = \frac{1 - x + 1 - y}{z}
$$
\textbf{Getting fancy with fraction identities}\\
Can be helpful for reducing tricky expressions:

$$
\frac{x}{y} + \frac{z}{a} = \frac{x}{y} \cdot \frac{n}{n} + \frac{z}{y} \cdot
\frac{n+k}{n+k}
$$
It's occurring to me I'm pretty rusty on my basic algebraic manipulations within
fractions. See the following example of simplifying an expression:

\begin{align*}
    3x^{3/2} - 9x^{1/2} + 6x^{-1/2}\\
    3x^{3/2} - 9x^{1/2} + \frac{6}{x^{1/2}}\\
    \frac{3x^2}{x^{1/2}} - \frac{9x}{x^{1/2}} + \frac{6}{x^{1/2}}\\
    \frac{3x^2 - 9x + 6}{x^{1/2}}\\
    \frac{3(x^2 - 3x + 2)}{x^{1/2}}\\
    3x^{-1/2}(x-3)(x-1)
\end{align*}

When separated by addition / subtraction, each term needs to be treated as its own thing.
But when separated by multiplication, terms are very pliable. For example the exponent in
$6x^{1/2}$ applies only to the $x$ not to the $6$. When I think about it, it's obvious
that that's so with something more ordinary like $6x^2$. Like $6x^2 \neq 36x^2$... \\

\textbf{Exponents and fractions}

\begin{align}
    \left( \frac{x}{y}\right) ^n = \frac{x^n}{y^n}\\
    \frac{x^m}{x^n} = x^{m-n}\\
    \left( \frac{n}{m}\right) ^{-2} = \frac{m^2}{n^2}\\
    a^{m/n} = \sqrt[m]{a^n}\\
    a^{-m/n} = \frac{1}{\sqrt[m]{a^n}}\\
    a^{-m/n} = (a^m)^{\frac{-1}{n}}
\end{align}

Rule (6) is super nifty and relies on the nested fraction multiplication rule (defined
below).\\

\emph{Example.}

In the diagnostic exercises we have the following expression and the solution which I
really like comes from Bing AI:

\begin{align*}
    16^{-3/4}\\
    (2^4)^{-3/4}\\
    2^{4/1 \cdot -3/4}\\
    2^{-3}\\
    \frac{1}{2^3}\\
    \frac{1}{8}
\end{align*}

Factoring out powers is a really neat way of simplifying things here.\\

More fraction-fu with radicals:

\begin{align*}
    2x(4-x)^{-1/2} - 3 \sqrt{4 - x} &= 0\\
    \frac{2x - 3\sqrt{4-x}}{\sqrt{4-x}} &= 0\\
    \frac{\sqrt{4-x}}{\sqrt{4-x}} \cdot \frac{2x - 3\sqrt{4-x}}{\sqrt{4-x}} &= 0\\
    \frac{2x\sqrt{4-x} - 3(4-x)}{4-x} &= 0\\
    \frac{4-x}{4-x} \cdot \frac{2x\sqrt{4-x} - 3(4-x)}{4-x} &= 0\\
    \frac{2x(4-x) - 3(4-x)^2}{(4-x)^2} &= 0\\
    \frac{(4-x)(2x - 3(4-x))}{(4-x)^2} &= 0\\
    \frac{2x-12 + 3x}{4-x} &= 0\\
    5x - 12 = 0\\
    x = \frac{12}{5}
\end{align*}

Serious fraction-fu.

\section{Exponents}

Negative bases:

\begin{align}
    (-3)^2 = 9\\
    -3^2 = -9\\
    (-3)^3 = -27\\
\end{align}

Multiplication and nested exponents:

\begin{align*}
    x^2 \cdot x^3 = x^5\\
    (x^2)^3 = x^6
\end{align*}

This $x^2 + x^3$ can't be reduced.

\section{Radicals}

\textbf{Product rule}

$$
\sqrt{AB} = \sqrt{A} \cdot \sqrt{B}
$$

\textbf{Quotient rule}

$$
\frac{\sqrt{A}}{\sqrt{B}} = \sqrt{\frac{A}{B}}
$$

\textbf{Simplifying}

$$
\sqrt{x^2y} = x\sqrt{y}
$$
\section{Inequalities}

One notion I forgot about is \emph{critical points}. This is a number which makes the
expression equal to zero or undefined. You want to find all critical points and then test
in the intervals between them to see whether the inequality holds. Here's an example from
the diagnostic list using an undefined critical point that I got wrong.

\begin{align*}
    \frac{2x-3}{x+1} \leq 1\\
\end{align*}

We note that one critical point is $-1$ (denominator). To find more we change the
inequality to an equality and solve.

\begin{align*}
    \frac{2x-3}{x+1} = 1\\
    2x-3 = x+1\\
    x = 4
\end{align*}

After testing the intervals in and around $-1,~4$ we find that the inequality only holds
in the middle, ie $(-1, 4]$.


\section{Things to Include}

\begin{itemize}
    \item Exponent rules when there are coefficients.
    \item Cancelling term rules in fractions
    \item Factoring cubic polynomials.
\end{itemize}


\end{document}
