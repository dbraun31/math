\chapter{Basic Properties of Numbers}

\begin{quote}
    To be conscious that you are ignorant is a great step to knowledge.
    \hfill -- Benjamin Disraeli
\end{quote}

We're going to consider twelve properties of numbers, and the first nine are concerned
with addition and multiplication.

\section{Addition}

\begin{definition}
    Regard addition as an operation which can be performed on a pair of numbers, $a + b$.
\end{definition}

He argues it might be intuitive to want to define addition over a series of numbers, but
that series can just be broken down into summing of different pairs. This reveals that the
order of the summation yields equivalent results, which gives our first property:

\begin{property}[Associative law for addition]
    If $a,~b,$ and $c$ are any numbers, then
    $$
    a + (b + c) = (a + b) + c
    $$
\end{property}

This generalizes straightforwardly to $a_1 + \ldots + a_n$. He notes that a reasonable
approach to showing this extension is outlined in Problem 24. 

\subsection{Properties involving zero}

\begin{property}[Existence of an additive identity]\label{prop:additiveidentity}
    If $a$ is any number, then 
    $$
    a + 0 = 0 + a = a.
    $$
\end{property}

\begin{property}[Existence of additive inverses]
    For every number $a$, there is a number $-a$ such that
    $$
    a + (-a) = (-a) + a = 0
    $$
\end{property}

We can now use these three properties to prove a simple assertion, which is that if a
number $x$ satisfies $a + x = a$, $x=0$.

\begin{alignat*}{3}
    &\text{If} \quad&  a + x &= a,\\
    &\text{then} \quad & (-a) + (a + x) &= (-a) + a = 0;\\
    &\text{hence} \quad & ((-a) + a) + x &= 0;\\
    &\text{hence} \quad & 0 + x &= 0\\
    &\text{hence} \quad & x &=0
\end{alignat*}

Ahhh that's pretty cool. He goes on to show that you can support basic algebraic maneuvers
with these three properties. 

\begin{property}[Commutative law for addition]
    If $a$ and $b$ are any numbers, then
    $$
    a + b = b + a
    $$
\end{property}

He goes on to say that commutative law doesn't hold for other types of relations (eg,
subtraction). But in order to have all algebra moves on the table, it's necessary to
introduce multiplication.

\section{Multiplication}

\begin{property}[Associative law of multiplication]
    If $a,~b$ and $c$ are any numbers, then
    $$
    a \cdot (b\cdot c) = (a \cdot b) \cdot c
    $$
\end{property}

\begin{property}[Existence of multiplicative identity]
    If $a$ is any number, then
    \begin{align*}
        a \cdot 1 = 1 \cdot a = a\\
        \text{Moreover,}~1 \neq 0
    \end{align*}
\end{property}

He's saying we have to list $1\neq 0$ because there's no way to prove based only on the
other properties---they would all hold if $0$ was the only number. 

\begin{property}[Existence of multiplicative inverses]\label{prop:multinverse}
    For every number $a \neq 0$, there is a number $a^{-1}$ such that
    $$
    a \cdot a^{-1} = a^{-1} \cdot a = 1
    $$
\end{property}

\begin{property}[Commutative law of multiplication]
    If $a$ and $b$ are any numbers, then
    $$
    a \cdot b = b \cdot a
    $$
\end{property}

Points out how necessary $a\neq 0$ is in Property \ref{prop:multinverse}, as there is no
number $0^{-1}$ satisfying $0 \cdot 0^{-1} = 1$. This is cool: just as how subtraction was
defined in terms of addition, division is defined in terms of multiplication. Note that
the usual form of an inverse (at least for an integer) is:

$$
a^{-1} = \frac{1}{a}
$$

So he says the symbol $a/b$ can be thought of as $a \cdot b^{-1}$. We can show that, as
long as $a\neq 0$, then for $a \cdot b = a \cdot c$, that $b = c$.

\begin{alignat*}{3}
    &\text{If}\quad& a\cdot b  &= 0~\text{and}~ a \neq 0,\\
    &\text{then}\quad&   a^{-1} \cdot (a \cdot b)  &= a^{-1} \cdot (a \cdot c);\\
    & \text{hence}\quad&  (a^{-1} \cdot a) \cdot b  &= (a^{-1} \cdot a) \cdot c;\\
    & \text{hence} \quad & 1 \cdot b &= 1 \cdot c;\\
    & \text{hence} \quad & b&=c.
\end{alignat*}

It also follows from Property \ref{prop:multinverse} that if $a \cdot b = 0$ then either
$a = 0$ or $b = 0$ (implied that this includes the possibility that they're both zero). We
can show that, if we know one of them \emph{isn't} zero, that the other one must be:

\begin{alignat*}{3}
    &\text{if} \quad& a \cdot b &= 0~\text{and}~ a \neq 0,\\
    & \text{then} \quad& a^{-1} \cdot (a \cdot b) &= 0;\\
    & \text{hence} \quad& (a^{-1} \cdot a) \cdot b &= 0;\\ 
    & \text{hence} \quad& 1 \cdot b &= 0; \\
    & \text{hence} \quad& b &=0.
\end{alignat*}

This concept of one or another variable must be equal to zero comes into play in the
familiar factoring of binomial situations (eg, $(x-1)(x-2) = 0$), where we know $x = 1$ or
$x = 2$.

This next property gives us tremendous ability to prove lots of things:

\begin{property}[Distributive law]\label{prop:distributive}
    If $a,~b,$ and $c$ are any numbers, then 
    $$
    a \cdot (b + c) = a \cdot b + a \cdot c
    $$
\end{property}


With Property \ref{prop:distributive}, we can now show the only case when $a - b = b-a$:

\begin{alignat*}{3}
    &\text{If}\quad& a - b &= b - a, \\
    & \text{then}\quad& (a-b) + b &= (b-a) + b = b + (b-a);\\
    & \text{then}\quad& a &= b + b -a;\\
    & \text{then}\quad& a + a &= (b+b-a)+a = b+b;\\
    & \text{Consequently}\quad& a \cdot(1+1)&= b \cdot (1+1),\\
    &\text{and therefore}\quad& a &=b.
\end{alignat*}

\subsection{More zero stuff}

With Property \ref{prop:distributive}, we can do more reckoning around zero, such as
showing that $a \cdot 0 = 0$. (Some of this upcoming logic gets pretty nifty). The proof:

\begin{align}
    a \cdot 0 &= a \cdot (0+0) \label{step3} \\
              &= a\cdot0 + a \cdot 0\\
    (a\cdot 0) - (a \cdot 0) &= a \cdot 0 + a \cdot 0 - a \cdot 0\\
    0 &= a \cdot 0
\end{align}

Step \ref{step3} is by Proposition \ref{prop:additiveidentity} (the additive identity).

Left off near bottom of page 7.
