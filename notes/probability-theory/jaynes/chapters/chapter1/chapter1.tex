\chapter{Plausible Reasoning}

\begin{quotation}
    The actual science of logic is conversant at present only with things either centrain,
    impossible, or entirely doubtful, none of which (fortunately) we have to reason on.
    Therefore the true logic for this world is the calculus of Probabilities, which takes
    account of the magnitude of the probability which is, or ought to be, in a reasonable
    man's mind.\\
    \hfill -- James Clerk Maxwell (1850)
\end{quotation}

\section{Deductive and plausible reasoning}

Opening with the example of a police officer witnessing a burglar alarm at night, a broken
window in a jewelry store, and a man with a mask walking away with a bag. The police
officer doesn't have complete information to conduct perfect deductive logical reasoning,
but the officer likely concludes that this person burglarized the store. More generally,
we live in a world of incomplete information, but we are constantly making plausibility
judgments about everyday phenomena.

The broad goal is to establish theorems and rules based on elementary---and nearly
inescapable---rules of rationality that will replace conflicting intuitive judgments and
\emph{ad hoc} procedures. He draws a distinction between (strong) deductive reasoning, and
(weaker) inductive (my term) reasoning.

\subsection{Deductive reasoning}

We can analyze deductive reasoning as repeated application of two strong syllogisms:

\begin{gather*}
    \text{if } A \text{ is true, then } B \text{ is true}\\
    \frac{A \text{ is true}}{\text{therefore, } B \text{ is true}}
\end{gather*}

\noindent and it's inverse

\begin{gather*}
    \text{if } A \text{ is true, then } B \text{ is true}\\
    \frac{B \text{ is false}}{\text{therefore, } A \text{ is false.}}
\end{gather*}

\noindent Thinking through even just this gets tricky. He emphasizes that these are
logical and not causal relationships, but I find it hard to distinguish. In the framing of
the premise, $B$ is clearly felt to be the consequence of $A$, and I think, logically,
that's how it should be thought of. Yes, he says $B$ is one of the consequences of $A$.

\subsection{Inductive reasoning}

In almost all situations, we need to fall back on weaker syllogisms:

\begin{gather*}
    \text{if } A \text{ is true, then } B \text{ is true}\\
    \frac{B \text{ is true}}{\text{therefore, } A \text{ becomes more plausible.}}
\end{gather*}
\\

\noindent This syllogism reveals that it's not correct to think of $A$ as being necessary
for $B$. It's just hard to think of what ``consequence'' means in such abstract terms.

\begin{gather*}
    \text{if } A \text{ is true, then } B \text{ is true}\\
    \frac{A \text{ is false}}{\text{therefore, } B \text{ becomes less plausible.}}
\end{gather*}
\\
\noindent He says the policeman makes even weaker reasoning

\begin{gather*}
    \text{if } A \text{ is true, then } B~\text{becomes more plausible}\\
    \frac{B \text{ is true}}{\text{therefore, } A \text{ becomes more plausible.}}
\end{gather*}
\\
\noindent I think this gets interpreted as the following

\begin{align*}
    & A \equiv~\text{The shop is getting burglarized by the man on the street}\\
    & B \equiv~\text{The alarm is going off, the man is wearing a mask and carrying a bag,
    etc.}
\end{align*}
\\
\noindent This feels like it's drawing a distinction between observable facts---which can
have certain true / false values---versus the more latent truth (eg, about a population),
which can never be known for certain.

He goes on to discuss how, even though this is a very weak form of reasoning, the
subjective plausibility for the police officer was likely very high. Which implies we need
some way of representing the \emph{degree of plausibility}. And he's also emphasizing the
importance of prior information / belief in constructing that sense of plausibility. We
conceal how complicated this type of reasoning is by calling it \emph{common sense}.

\subsection{Logical vs. causal connections}

He warns pretty much immediately about the distinction between logical and causal
connection, and that we are primarily concerned with the former. He gives the following
example events for $A$ and $B$

\begin{align*}
    & A \equiv~\text{it will start to rain by 10 AM at the latest};\\
    & B \equiv~\text{the sky will become cloudy before 10 AM.}
\end{align*}

\noindent This makes it clear that $B$ being true is only a \emph{logical consequence} of
$A$ being a true---it is a thing that follows logically. In this example, $A$ is clearly
not the cause of $B$. The proper causal connection follows \emph{certainty} (ie, rain
$\implies$ clouds) and not the uncertain causal connection (ie, clouds $\implies$ rain).

\subsection{A hopeful note about plausible reasoning}

He ends this section by pointing out that, even though in deductive reasoning the
conclusions have just as much certainty as the premises, he will show how the same can be
accomplished for inductive reasoning based on plausibility. The reliability of the
conclusions changes as one goes through the steps of reasoning, but he argues that you can
arrive at conclusions that can approach the certainty of deductive reasoning. \footnote{He
    interestingly points out that even mathematicians rely on inductive reasoning to
    arrive at theorems and results. The process of coming up with speculations is often
highly inductive.}

\section{Analogies with physical theories}
