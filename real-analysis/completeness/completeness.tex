\documentclass{article}
\usepackage{hyperref}
\usepackage{graphicx}
\usepackage{amsmath}
\usepackage{amsfonts}
\usepackage{amsthm}
\usepackage{enumerate}
\usepackage[indent=20pt]{parskip}
\numberwithin{equation}{subsection}

\theoremstyle{definition}
\newtheorem{definition}{Definition}
\newtheorem{example}{Example}

\begin{document}



\title{Understanding Analysis - Chapter 1 Notes}
\author{Dave Braun}
\maketitle

\section{The Real Numbers}
\setcounter{subsection}{2}
\subsection{The Axiom of Completeness}

What is $\mathbb{R}$? The author talking about challenges around providing precise
definitions, and at some point one has to draw an arbitrary line and accept that as a
starting point. Detailing a bit of the history, saying that it was an intuitive
understanding of $\mathbb{R}$ that really led the way, followed by methods for rigorously
constructing $\mathbb{R}$ from the set of rational numbers $Q$.

\subsubsection{An Initial Definition for R}

$\mathbb{R}$ is an extension of $Q$, meaning that every element in $\mathbb{R}$ has an
additive inverse and every nonzero element has a multiplicative inverse. $\mathbb{R}$ is a
\emph{field}, where addition and multiplication are commutative, associative, and the
distributive property holds. This gives us algebra and logical orderings, such as "If $a <
b$ and $c > 0$, then $ac < bc$". Finally, we need a way of insisting that $\mathbb{R}$
does not contain the gaps in its number line that $Q$ contain. \\

\textbf{Axiom of Completeness.} \emph{Every nonempty set of real numbers that is bounded
above has a least upper bound.}


\subsubsection{Least Upper Bounds and Greatest Lower Bounds}

Beginning with definitions.

\begin{definition}
    A set $A \subseteq \mathbb{R}$ is \emph{bounded above} if there exists a number $b \in
    \mathbb{R}$ such that $a \leq b$ for all $a \in A$. The number $b$ is called an
    \emph{upper bound} for $A$. \\
    Similarly, the set $A$ is \emph{bounded below} if there exists a \emph{lower bound} $l
    \in \mathbb{R}$ satisfying $l \leq a$ for every $a \in A$.

\end{definition}

\begin{definition}
    A real number $s$ is the \emph{least upper bound} for a set $A \subseteq \mathbb{R}$
    if it meets the following two criteria:

    \begin{enumerate}[(i)]
        \item $s$ is an upper bound for $A$.\\
        \item if $b$ is any upper bound for $A$, then $s \leq b$.
    \end{enumerate}
\end{definition}

Least upper bound also referred to as the \emph{supremum} of the set $A$, also $s =
\text{lub} A$. This text will use $s = \text{sup} A$. $s = \text{inf} A$ will be used to
denote lower bound.

Okay so the upper and lower bounds are just the highest and lowest elements in the set,
because, for highest: $a \leq b$ for all $a \in A$ and all $b \in \mathbb{R}$.\\

Oh he goes on to show how this intuition isn't always true.

\begin{example}
    
    $$
    A = \left\{ \frac{1}{n}: n \in N \right\} = \left\{ 1, \frac{1}{2}, \frac{1}{3}, \ldots \right\}.
    $$

   The set $A$ is bounded above and below. The upper bound is $1$. The lower bound is
    more difficult... it would be $\frac{1}{\infty}$ or $0$.

    A lesson to note here is that the sup and inf of a set are not always elements of that
set.

\end{example}


\begin{definition}
    A real number $a_0$ is a \emph{maximum} of the set $A$ if $a_0$ is an element of $A$
    and $a_0 \geq a$ for all $a \in A$. Similarly, a number $a_1$ is a \emph{minimum} of
    $A$ if $a_1 \in A$ and $a_1 \leq a$ for every $a \in A$.
\end{definition}


\begin{example}
    To further illustrate the point between bounds and maxima / minima, consider the open
    interval:

    $$
    (0, 2) = \{x \in \mathbb{R}: 0 < x < 2\},
    $$
    and the closed interval
    $$
    [0, 2] = \{x \in \mathbb{R}: 0 \leq x \leq 2\}.
    $$
    Both of these sets are bounded in both directions, but only one set (the closed
    interval) has a maximum. There is no element in the open interval that is the maximum
    of the set.

    Axiom of Completeness asserts that every nonempty bounded set has a least upper bound.
    \\

    An axiom is meant to be a statement that's so clear or intuitive that it can be
    accepted on its face and needs no proof.

\end{example}

Left off around p. 17.

\end{document}
