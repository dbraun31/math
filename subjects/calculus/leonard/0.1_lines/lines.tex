\documentclass{article}
\usepackage{graphicx}
\usepackage{amsmath}
\usepackage{enumerate}
\usepackage{amsfonts}

\begin{document}

\title{Lecture 0.1: Lines, Angle of Inclination, and the Distance Formula}
\author{Professor Leonard}
\maketitle

This lecture is going over some super basic things from algebra and maybe some less basic
things from trig. The idea here is to take notes on some of the basic things that I forgot
about.


\subsection{Lines}
Lines are infinite, straight, have infinite numbers of points, and have a slope. Need at
least two points to define a slope:

\begin{equation}
    m = \frac{y_2 - y_1}{x_2 - x_1}
\end{equation}

Point slope equation:

\begin{equation}
    y - y_1 = m(x - x_1)
\end{equation}

Parallel lines have the same slope.

General form: 

\begin{equation}
    Ax + By = C
\end{equation}
A, B, and C are i n t e g e r s.\\

Perpendicular lines have negative reciprocal slopes. Eg, reciprocal of $-3/2$ is $2/3$.

\subsection{Angles of Inclination}

\textit{Angle of inclination is the angle that a line makes with the x axis.}\\

Bringing in sine, cosine, tangent. Take a line, make a triangle where the sides are change
in y and x, tangent of an angle equals opposite (length) over adjacent (length).\\

Remember, angles can be expressed in degrees or radians. $degrees \cdot \pi/180 =
radians$.\\

Remember about \textit{rationalizing the denominator}. You don't want to have a radical in
the denominator---need to multiply to get it out. Eg, $\frac{1}{\sqrt{3}} =
\frac{\sqrt{3}}{3}$.\\

Distance:\\

\begin{equation}
    d(p,q) = \sqrt{\sum_{i=1}^{N}{(p_i - q_i)^2}}
\end{equation}

$d(p,q)$ reflects the distance between points $p$ and $q$ in $N$ dimensional space.


\end{document}

