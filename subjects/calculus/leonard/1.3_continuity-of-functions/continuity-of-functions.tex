\documentclass{article}
\usepackage{graphicx}
\usepackage{float}
\usepackage{amsmath}
\usepackage{enumerate}
\usepackage{amsfonts}
\usepackage{tikz}
\usepackage{amsthm}
\usepackage{mdframed}


\newmdtheoremenv{definition}{Definition}
\newmdtheoremenv{theorem}{Theorem}

\setlength{\parindent}{0pt}

\begin{document}

\title{Lecture 1.3: Continuity of Functions}
\author{Professor Leonard}
\maketitle

\section{Defining Continuity}

\begin{definition}
A function is \underline{continuous} if it has no holes,
breaks or asymptotes. Mathematically, a function is continuous at point $c$ if:
\begin{enumerate}
	\item $f(c)$ is defined.
	\item $\lim_{x\to c} f(x)$ must exist.
	\item $\lim_{x\to c} f(x) = f(c)$
\end{enumerate}
\end{definition}

It's important to be specific about \emph{where} a function is continuous or not. For
example, a hole in a function is referred to as a \emph{removable discontinuity}.\footnote{I got shook a bit initially thinking point $c$ was in the range of function $f$ but it's actually in the \emph{domain}. Remember, $\lim_{x\to c} f(x)$ means $x$ (the input) is approaching point $c$ (in the domain).\emph{Edit:} I guess actually a \emph{point} $(x, y)$ consists of both domain and range.}

\begin{definition}
A \underline{removable discontinuity} is a discontinuity that could be filled in with a
single point.
\end{definition}

\begin{definition}
A \underline{jump discontinuity} is a visual 'jump' in the function as the function
approaches a point (ie, the limit doesn't exist).
\end{definition}

\subsection{Examples}

Are the following functions continuous at $x=2$?

\begin{align*}
    f(x) &= \frac{x^2 - 4}{x-2}\\
    g(x) &= \begin{cases}
        \frac{x^2- 4}{x-2} & \text{if } x \neq 2\\
        3 & \text{if } x = 2
    \end{cases}\\
        h(x) &= \begin{cases}
            \frac{x^2 - 4}{x-2} & \text{if } x \neq 2\\
            4 & \text{if } x = 2
        \end{cases}
\end{align*}

Case 2 is worth highlighting. We need to check whether $g(2) = \lim_{x\to 2} g(x)$.

\begin{align*}
    \lim_{x\to 2} \frac{x^2-4}{x-2} &= \lim_{x\to 2} \frac{(x-2)(x+2)}{x-2}\\
                                    &= \lim_{x\to2} x+2\\
    g(2) = 3 &\neq 4
\end{align*}

The function is not continuous at $2$, because $g(2) \neq \lim_{x\to2} g(x)$.

    
\section{Endpoints}

We need to establish a few facts in order to work with continuity of endpoints.

\begin{theorem}
    If a function $f$ is continuous at \underline{every} point between $a$ and $b$, then
    $f$ is continuous on the open interval $(a, b)$.
\end{theorem}

Also need to check whether a function is continuous at a point when approaching from
either side.

\begin{theorem}
    Continuous from the left at point $c$:
    $$
    \lim_{x\to c^-} f(x) = f(c)
    $$
    Continuous from the right at point $c$:
    $$
    \lim_{x\to c^+} f(x) = f(c)
$$
\end{theorem}

\subsection{Examples}

Prove the following function is continuous at $[-4, 4]$.

$$
f(x) = \sqrt{16-x^2}
$$

I worked this out on the board, but the jist of it is to first check the open interval
$(-4, 4)$ for domain problems.\footnote{This was kind of an unsatisfying proof. You just
    sort of intuitively check the interval and if it's okay then you declare it's all
good.} Then, for the one sided limits at the endpoints, you actually are evaluating the
limit of the function at that endpoint \emph{from} the side (pos/neg) that the function is
coming from. For example, to evaluate the $-4$ endpoint, you check 

$$
\lim_{x\to-4^+} f(x)
$$

I think unless your limit is at infinity or is some other type of domain problem, then you
can just plug it in and it should be fine. Otherwise I guess you need to do a sign
analysis test.
 
Left off at 35:45.

\end{document}

