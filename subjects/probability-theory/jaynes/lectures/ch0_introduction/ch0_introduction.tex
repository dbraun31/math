\documentclass{article}
\usepackage{graphicx}
\usepackage{amsmath}
\usepackage{enumerate}
\usepackage{amsfonts}
\usepackage[hidelinks]{hyperref}
\usepackage{xcolor}

\begin{document}

\title{Course Introduction}
\author{Aubrey Clayton}
\maketitle

See lecture \href{https://www.youtube.com/watch?v=rfKS69cIwHc&list=PL9v9IXDsJkktefQzX39wC2YG07vw7DsQ_&index=1}{\textcolor{blue}{here}}.

\section{History of Probability Theory}

Mathematical notions of probability are thought to have originated from games and
gambling. Dates back to letters exchanged between Pascal and Fermat trying to figure out
the fair odds of throwing dice. Kolmogorov was the first to formalize probability (1933),
which is a subset of measure theory.

He introduced probability by talking about the linguistic origins. He's arguing that it's
pretty baked into our language (eg, when casually talking about odds of something
happening). He's also pointing to the idea that the term \emph{probability} is a bit
ambiguous as to whether we're referring either to (i) a formal 'law' or way of making
decisions or setting beliefs, or (ii) a way of describing frequencies of events unfolding over many
observations. 

\section{History of Statistics}

The term \emph{statistics} originally referred to "of the state", or estimations that were
of interest to the state. Understood that these estimations were subject to error such as
measurement error or sampling bias. Another difficult question was how to combine
different observations of things like astronomical data. Developmed most prominently by
Laplace. These guys borrowed from probability theory to more accurately describe
measurement errors---the language here is very much frequentist. A nice way of summing up
statistics vs. probabilities is one is trying to quantify measurement frequencies and the other
is trying to quantify frequencies of events.

Pearson and Neyman were working in parallel to and or competition with Fisher to develop a
way of assesesing measurement error, and together developed the so-called orthodox
statistics.

\section{General Discussion}

\textbf{Interpretations of probability.}\\

He quickly introduces Bayes, which is represented on both sides of his little
probability / statistics chart.

He makes the interesting note that Kolmogorov's work doesn't really adjudicate between
frequentist / Bayesian. It's just a set of rules for how probability is manipulated once a
sample space is defined---it doesn't really give you a way to think about what probability
really is and how to conceptually map it on to the real world.

When the concept of probability got applied beyond games of chance (some speculation), it
really takes on this meaning of \emph{degrees of belief}. A concern then is that degrees
of belief are inherently subjective. \\

\textbf{Scientific practice.}\\

Hm okay now he's saying that Kolmogorov's axioms really only make sense when you're
considering frequencies and ratios---they wouldn't really make sense if you're talking
about degrees of belief. Jaynes reconciles this tension between probability and statistics
by formalizing probability as an extension of logic, but in cases where we're reasoning
with incomplete knowledge (ie, not deductions). Jaynes derives the Kolmogorov axioms from
more fundamental logical reasoning.  In frequentist statistics, conditional probability is
kind of a weird offshoot, and Bayes rule comes from that. But when reasoning from
Aristotelian logic, Bayes theorem comes very centrally. 

Jaynes launches a huge critique of the Fisher frequentist methods. Frequentist logic
starts from some null hypothesis, constructing a hypothetical distribution of summary
statistics conditional on that hypothesis, and contrasts the observed data against that.
Jaynes argues we need to start with the data as a given, and talk about the probability of
the hypothesis given the observed data. Ie, instead of going hypothesis to data, we're
going data to hypothesis.

Concludes by foreshadowing that Jaynes uses his Aristotelian logic and Bayes rule to
dismantle the work of the last 200 years of orthodox statistics, and apparently Jaynes
makes personal insults against Fisher throughout the book. Can't wait!

\end{document}

