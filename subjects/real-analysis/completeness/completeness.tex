\documentclass{article}
\usepackage{hyperref}
\usepackage{graphicx}
\usepackage{amsmath}
\usepackage{amsfonts}
\usepackage{amsthm}
\usepackage{enumerate}
\usepackage[indent=20pt]{parskip}
\numberwithin{equation}{subsection}

\theoremstyle{definition}
\newtheorem{definition}{Definition}
\newtheorem{example}{Example}
\newtheorem{lemma}{Lemma}
\newtheorem{theorem}{Theorem}

\begin{document}



\title{Understanding Analysis - Chapter 1 Notes}
\author{Dave Braun}
\maketitle

\section{The Real Numbers}
\setcounter{subsection}{2}
\subsection{The Axiom of Completeness}

What is $\mathbb{R}$? The author talking about challenges around providing precise
definitions, and at some point one has to draw an arbitrary line and accept that as a
starting point. Detailing a bit of the history, saying that it was an intuitive
understanding of $\mathbb{R}$ that really led the way, followed by methods for rigorously
constructing $\mathbb{R}$ from the set of rational numbers $Q$.

\subsubsection{An Initial Definition for R}

$\mathbb{R}$ is an extension of $Q$, meaning that every element in $\mathbb{R}$ has an
additive inverse and every nonzero element has a multiplicative inverse. $\mathbb{R}$ is a
\emph{field}, where addition and multiplication are commutative, associative, and the
distributive property holds. This gives us algebra and logical orderings, such as "If $a <
b$ and $c > 0$, then $ac < bc$". Finally, we need a way of insisting that $\mathbb{R}$
does not contain the gaps in its number line that $Q$ contain. \\

\textbf{Axiom of Completeness.} \emph{Every nonempty set of real numbers that is bounded
above has a least upper bound.}


\subsubsection{Least Upper Bounds and Greatest Lower Bounds}

Beginning with definitions.

\begin{definition}
    A set $A \subseteq \mathbb{R}$ is \emph{bounded above} if there exists a number $b \in
    \mathbb{R}$ such that $a \leq b$ for all $a \in A$. The number $b$ is called an
    \emph{upper bound} for $A$. \\
    Similarly, the set $A$ is \emph{bounded below} if there exists a \emph{lower bound} $l
    \in \mathbb{R}$ satisfying $l \leq a$ for every $a \in A$.

\end{definition}

\begin{definition}
    A real number $s$ is the \emph{least upper bound} for a set $A \subseteq \mathbb{R}$
    if it meets the following two criteria:

    \begin{enumerate}[(i)]
        \item $s$ is an upper bound for $A$.\\
        \item if $b$ is any upper bound for $A$, then $s \leq b$.
    \end{enumerate}
\end{definition}

Least upper bound also referred to as the \emph{supremum} of the set $A$, also $s =
\text{lub} A$. This text will use $s = \text{sup} A$. $s = \text{inf} A$ will be used to
denote lower bound.

Okay so the upper and lower bounds are just the highest and lowest elements in the set,
because, for highest: $a \leq b$ for all $a \in A$ and all $b \in \mathbb{R}$.\\

Oh he goes on to show how this intuition isn't always true.

\begin{example}
    
    $$
    A = \left\{ \frac{1}{n}: n \in N \right\} = \left\{ 1, \frac{1}{2}, \frac{1}{3}, \ldots \right\}.
    $$

   The set $A$ is bounded above and below. The upper bound is $1$. The lower bound is
    more difficult... it would be $\frac{1}{\infty}$ or $0$.

    A lesson to note here is that the sup and inf of a set are not always elements of that
set.

\end{example}


\begin{definition}
    A real number $a_0$ is a \emph{maximum} of the set $A$ if $a_0$ is an element of $A$
    and $a_0 \geq a$ for all $a \in A$. Similarly, a number $a_1$ is a \emph{minimum} of
    $A$ if $a_1 \in A$ and $a_1 \leq a$ for every $a \in A$.
\end{definition}


\begin{example}
    To further illustrate the point between bounds and maxima / minima, consider the open
    interval:

    $$
    (0, 2) = \{x \in \mathbb{R}: 0 < x < 2\},
    $$
    and the closed interval
    $$
    [0, 2] = \{x \in \mathbb{R}: 0 \leq x \leq 2\}.
    $$
    Both of these sets are bounded in both directions, but only one set (the closed
    interval) has a maximum. There is no element in the open interval that is the maximum
    of the set.

    Axiom of Completeness asserts that every nonempty bounded set has a least upper bound.
    \\

    An axiom is meant to be a statement that's so clear or intuitive that it can be
    accepted on its face and needs no proof.

\end{example}

\begin{example}
    Consider the set:
    $$
    S = \{r \in Q: r^2 < 2 \}
    $$
    There are plenty of possible upper bounds, anything $b \geq 2$ will do. But we can't
    find a least upper bound where $b \in Q$, because the least upper bound should be $r =
    \sqrt{2}$, which is irrational. 

\end{example}

\begin{example}
    Let $A \subseteq \mathbb{R}$ be nonempty and bounded above, and let $c \in
    \mathbb{R}$. Define the set $c + A$ by

    $$
    c + A = \{ c + a: a \in A \}
    $$
    Then $\text{sup}(c + A) = c + \text{sup}A$.\\

    Need to verify the two aspects of least upper bound definition, namely that $s$ is an
    upper bound for $A$, and that if $b$ is any upper bound for $A$, then $s \leq b$.\\

    (Not looking at the explanation). I guess we need to show that $c + \text{sup}A$
    qualifies as an upper bound of $c + A$. So $c + \text{sup}A \geq a + c$ for all $a \in
    A$. I can propose some $s = \text{sup}A$, such that $s \geq a$ for all $a \in A$,
    meaning it's necessarily true that $s + c \geq a + c$. I think that proves the first
    component of the definition. \\

    Next is to show that for any $b \geq A + c$ that $s \leq b$. (looking at explanation)
    then $b - c \geq A$, so $b-c$ is an upper bound on $A$. Because $s$ is the least upper
    bound, we can write $s \leq b - c$, which can be rewritten as $s + c \leq b$, thus
    proving part two.

\end{example}

\begin{lemma} 
    Assume $s \in \mathbb{R}$ is an upper bound for a set $A \subseteq \mathbb{R}$. Then,
    $s = \text{sup}~A$ if and only if, for every choice of $\epsilon > 0$, there exists an
    element $a \in A$ satisfying $s - \epsilon < a$.

    \begin{proof}
       The idea is given that $s$ is an upper bound on set $A$, it is the least upper
       bound if and only if any number less than $s$ is in set $A$ and is not an upper
       bound.\\

       Prove it forwards, meaning that if $s = \text{sup}~A$ then for some arbitrarily
       chosen $\epsilon > 0$, there should be some $a \in A$ where $s - \epsilon < a$. We
       can note how $s - a < s$, and if $s = \text{sup}~A$, then anything less than $s$ is
       not an upper bound, and there should be some $a \in A$ where $a > s - \epsilon$
       (because otherwise $s - \epsilon$ would be an upper bound).

       Proving it backwards. Which means that if there exists some $a \in A$ which, for every
       choice $\epsilon > 0$, satisfies $s - \epsilon < a$, then $s = \text{sup}~A$. I can
       observe that $s < a + \epsilon$.\\

       (Their solution). Assume $s$ is an upper bound with the property that no matter how
       $\epsilon > 0$ is chosen, $s - \epsilon$ is no longer an upper bound for $A$.
       Notice that what this implies is that if $b$ is any number less than $s$, then $b$
       is not an upper bound. (Just let $\epsilon = s - b$). To prove that $s =
       \text{sup}~A$, we must verify part (ii) of the definition (which is that if $b$ is
       any upper bound, that $s \leq b$.) Because we have just argued that any number
       smaller than $s$ cannot be an upper bound, it follows that if $b$ is some other
       upper bound for $A$, then $s \leq b$.\\

       Struggling to follow this one, but okay.

    \end{proof}
\end{lemma}

\subsubsection*{Exercises}

\begin{enumerate}
    \item
    \begin{enumerate}
        \item Write a formal definition in the style of Definition 1.3.2 for the
            \emph{infimum} or \emph{greatest lower bound} of a set.\\

        \begin{definition}
            $i$ is the greatest upper bound of set $A \subseteq \mathbb{R}$ if:\\
            \begin{enumerate}
                \item $i$ is a lower bound on set $A$\\
                \item For any other lower bound $b$ on set $A$, $i \geq b$.
            \end{enumerate}
        \end{definition}

        \item Now, state and prove a version of Lemma 1.3.8 for greatest lower bounds.
            \begin{lemma}
                Assume $i \in \mathbb{R}$ is a lower bound on set $A \subseteq \mathbb{R}$. $i =
                \text{inf}~A$ if and only if for every choice of $\epsilon > 0$, $i +
                \epsilon \geq a$ for some $a \in A$.
                \begin{proof}
                    Okay so let's prove it forwards. Which is to say if $i =
                    \text{inf}~A$, then for every $\epsilon > 0$, $i + \epsilon \geq a$
                    for some  $a \in A$. If $i = \text{inf}~A$, then $i + a > i$. Since
                    any number plus $i$ can't be a lower bound if $i$ is the greatest
                    lower bound, then we know that if any number is added to $i$ it will
                    no longer be a lower bound and there should be some $a \in A$ where $i
                    + \epsilon \geq a$.\\

                    Proving it backwards. Assume $i \in \mathbb{R}$ is a lower bound on
                    set $A \subseteq \mathbb{R}$. If for every choice of $\epsilon > 0$,
                    $i + \epsilon \geq a$ for some $a \in A$, then $i = \text{inf}~A$.
                    Intuitively, it's clear that, if $i$ is a lower bound, and any nudge
                    in the positive direction causes $i$ to no longer be a lower bound,
                    then $i$ must be the greatest lower bound. I can also propose an
                    arbitrary lower bound $b$. Part (ii) of the definition requires that
                    $i \geq b$. Then I guess this is self evident because anything greater
                    than $i$ would no longer be a lower bound... ? But I struggle with
                    that step because $i = \text{inf}~A$ is the conclusion we're trying to
                    prove here, but the solution kind of relies on that as a premise to
                    support the claim $i \geq b$. Puzzling.
                    
                \end{proof}
            \end{lemma}

    \end{enumerate}
    \item Give an example of each of the following, or state that the request is
        impossible.\\
        \begin{enumerate}
            \item A set $B$ with $\text{inf}~B \geq \text{sup}~B$.\\
                I think the only thing that satisfies this is a set with one element (eg,
                $B = \{3\}$. Such that $\text{inf}~B = 3 = \text{sup}~B$.

            \item A finite set that contains its infimum but not its supremum.\\

                Hmmm. Maybe a set with one closed interval and one open? $A = [0, 4)$.\\

                Okay the answer is that this is impossible because finite sets contain
                their upper and lower bounds.

            \item A bounded subset of $Q$ that contains its supremum but not its
                infimum.\\

                My intuition is to answer this one similarly to the last, which suggests
                that it's probably wrong. The only question is whether the set with an
                open interval is technically rational or not, because you can't really
                pick out a number on the rational number line to end the set. \\

                Let $B = \{r \in Q | 1 < r \leq 2\}$ we have $\text{inf}~B = 1 \notin B$
                and $\text{sup}~B = 2 \in B$.

        \end{enumerate}

    \item 
        \begin{enumerate}
            \item Let $A$ be nonempty and bounded below, and define $B = \{b \in R: b
                \text{is a lower bound for} A\}$. Show that $\text{sup}~B =
                \text{inf}~A$.\\

                I'm struggling with this one because in order for $\text{sup}~B =
                \text{inf}~A$ we'd need to show that $A$ and $B$ perfectly overlap at
                their upper and lower bounds, and I can't see that from the definitions.

                The solution just appeals to the definition and says it's self evident.
                But all we know is that some element of $B$, $b$, is a lower bound of $A$.
                We could have $A \cap B \neq \emptyset$ meaning that $\text{sup}~B >
                \text{inf}~A$. Or $A \cap B = \emptyset$ such that $\text{sup}~B <
                    \text{inf}~A$. I reject the question.

            \item Use (a) to explain why there is no need to assert that greatest lower
                bounds exist as part of the Axiom of Completeness.

                It seems like it should just be the inverse of the definition. Every non
                empty set of real numbers that is bounded below has a greatest lower
                bound. Even some set set $A \subseteq (3, 5]$, the greatest lower bound is
                is irrational but is still a real number.

        \end{enumerate}
            \item Let $A_1, A_2, A_3, \ldots$ be a collection of nonempty sets, each of
                which is bounded above.
        \begin{enumerate}
            \item Find a formula for $\text{sup}~(A_1 \cup A_2)$. Extend this to
                $\text{sup}~(\bigcup_{k=1}^n A_k)$.

            \begin{align*}
                \text{sup}~(A_1 \cup A_2) = \text{max}(\text{sup}(A_1),~\text{sup}(A_2))\\
                \text{sup}~\left(\bigcup_{k=1}^n A_k\right) = \text{sup}~\{\text{sup}~A_k | k = 1,
                \ldots, n \}
            \end{align*}

        \item Consider $\text{sup}(\bigcup_{k=1}^{\infty} A_k)$. Does the formula in (a)
            extend to the infinite case?

            No, because set $\bigcup_{k=1}^{\infty} A_k$ might be unbounded, for example
            with $A_n = [n, n+1]$.

        \end{enumerate}

        Skipping ahead.

\end{enumerate}

\subsection{Consequences of Completeness}

\begin{theorem} \textbf{(Nested Interval Property)}. For each $n \in N$, assume we are
    given a closed interval $I_n = [a_n, b_n] = \{x \in \mathbb{R} : a_n \leq x \leq b_n
    \}$. Assume also that each $I_n$ contains $I_{n+1}$. Then, the resulting nested
    sequence of closed intervals
    \begin{equation*}
        I_1 \supseteq I_2 \supseteq I_3 \supseteq I_4 \supseteq \ldots
    \end{equation*}
    has a nonempty intersection; that is, $\bigcap_{n=1}^{\infty} I_n \neq \emptyset$.

    In human terms. You start with some interval $I_n = [a_n, b_n]$ in which there is some
    element $x$. Now you propose infinite subset intervals of $I_n$ just like closing in
    on themselves. The idea is to show that, even in those infinitely closing intervals,
    there is some $a_n \leq x \leq b_n$ such that there exists some intersection of all of
    those intervals $\bigcap_{k=1}^{\infty} I_n \neq \emptyset$.

\end{theorem}

\end{document}


